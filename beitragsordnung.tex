\documentclass[12pt,paper=a4,ngerman]{scrreprt}

\usepackage{babel}
\usepackage[T1]{fontenc}
\usepackage[utf8]{inputenc}
\usepackage{hyperref}
\usepackage{eurosym}
\usepackage{xspace}
\usepackage{chngcntr}
\usepackage{tocloft}
\usepackage[sfdefault,scaled=.85]{FiraSans}
\usepackage{newtxsf}
\usepackage{microtype}
\usepackage{cleveref}
\cftsetindents{section}{0em}{3em}

\author{Der HedgeDoc e.V.}

\hypersetup{colorlinks, linkcolor=black}

\renewcommand*\euro{\officialeuro\xspace}
\renewcommand\thesection{\S{\arabic{section}}}
\renewcommand\theenumi{\arabic{enumi}}
\renewcommand\theenumii{\alph{enumii}}

\labelformat{enumii}{\thesection.\theenumi.\theenumii)}
\labelformat{enumi}{\thesection.\theenumi.}

\addtocounter{secnumdepth}{1}


\title{Beitragsordnung des HedgeDoc e.V.}
\date{Stand: 23.02.2022 - Version: 1.0}
\begin{document}
	\counterwithout*{subsection}{section}
	\maketitle
	%\newpage
	
	\textbf{Änderungshistorie}
	\begin{description}
		\item
			Version 1: Beschlossen auf der Gründungsversammlung
	\end{description}
	\newpage
	
	\section{Grundsatz}
    	Diese Beitragsordnung ist nicht Bestandteil der Satzung. Sie regelt die Beitragsverpflichtungen der Mitglieder sowie die Gebühren und Umlagen. Sie kann nur von der Mitgliederversammlung des Vereins geändert werden.
	
	\section{Beschlüsse}
	\begin{enumerate}
	    \item
    	    Die Mitgliederversammlung beschließt die Höhe des Beitrags, die Aufnahmegebühr und Umlagen.
        \item 
    	    Die Beitragssätze gelten jeweils ab dem ersten Tag des auf den Beschluss folgenden Geschäftsjahres.
    \end{enumerate}
    	    
    \section{Beiträge}
    \begin{enumerate}
        \item
            Der Beitrag beträgt für jedes Mitglied pro Jahr mindestens 24 €. Auf freiwilliger Basis kann auch ein höherer Beitrag entrichtet werden.
        \item
            \label{sec:beiträge:mahngebühren}
            Die Mahngebühren für verspätete Entrichtung des Mitgliedsbeitrags betragen 5 € pro angefangenen Monat.
        \item
            Der Mitgliedsbeitrag deckt keine Kosten (z. B. Kursgebühren, Eintrittsgelder usw.) für Sonderveranstaltungen des Vereins ab.
    \end{enumerate}
    	    
	\section{Zahlung}
	    \begin{enumerate}
            \item
                Die Zahlung der Beiträge erfolgt in der Regel im Bankeinzugsverfahren.
            \item
                Der Jahresbeitrag ist zum ersten Tag des Geschäftsjahres zu entrichten. Mitglieder, die dem Verein kein Lastschriftmandat erteilt haben, erhalten eine Rechnung, die innerhalb von 14 Tagen nach Erhalt bezahlt werden muss. Bei verspäteter Beitragszahlung werden Mahngebühren erhoben, die sich aus \ref{sec:beiträge:mahngebühren} ergeben.
            \item
                Mitglieder, die dem Verein neu beitreten, zahlen im Beitrittsjahr jeweils 1/12 des Jahresbeitrags pro Monat ihrer Mitgliedschaft. Der Monat, in dem das Mitglied dem Verein beigetreten ist, wird nicht mitgerechnet.
	    \end{enumerate}
	        
\end{document}
