\documentclass[12pt,paper=a4,ngerman]{scrreprt}

\usepackage{babel}
\usepackage[T1]{fontenc}
\usepackage[utf8]{inputenc}
\usepackage{hyperref}
\usepackage{eurosym}
\usepackage{xspace}
\usepackage{chngcntr}
\usepackage{tocloft}
\usepackage[sfdefault,scaled=.85]{FiraSans}
\usepackage{newtxsf}
\usepackage{microtype}
\usepackage{cleveref}
\cftsetindents{section}{0em}{3em}

\author{Der HedgeDoc Verein}

\hypersetup{colorlinks, linkcolor=black}

\renewcommand*\euro{\officialeuro\xspace}
\renewcommand\thesection{\S{\arabic{section}}}
\renewcommand\theenumi{\arabic{enumi}}
\renewcommand\theenumii{\alph{enumii}}

\labelformat{enumii}{\thesection.\theenumi.\theenumii)}
\labelformat{enumi}{\thesection.\theenumi.}

\addtocounter{secnumdepth}{1}


\title{Satzung des HedgeDoc Verein}
\date{Stand: 17.03.2022 - Version: 1.0.0}
\begin{document}
	\counterwithout*{subsection}{section}
	\maketitle
	\newpage
	
	\textbf{Änderungshistorie}
	\begin{description}
		\item
			Version 1: Beschlossen auf der Gründungsversammlung am 17.03.2022
	\end{description}
	\newpage
	
	\tableofcontents
	\newpage
	
	\addsec{Präambel}
        Freie Software im Sinn dieser Satzung sind Computerprogramme, die von Urheber:innen in nicht rückholbarer Weise der Allgemeinheit quelloffen oder ausführbar zur Verfügung gestellt werden. Die Urheber:innen gewähren Dritten dabei die Freiheit, das Programm für jeden Zweck einsetzen zu dürfen; untersuchen zu dürfen, wie das Programm funktioniert und es den eigenen Bedürfnissen anzupassen; Kopien für Andere machen zu dürfen; und das Programm verbessern zu dürfen.
        
        Entwicklung freier Software im Sinn dieser Satzung umfasst die Erforschung und Ausarbeitung der theoretischen Grundlagen und Konzepte sowie deren Erprobung durch Programmierung und Test freier Software, welche diese Konzepte und Grundlagen realisiert.
        
        HedgeDoc ist ein internetbasiertes Programm für Zusammenarbeit an Dokumenten in Echtzeit und wird als freie Software entwickelt und der Allgemeinheit unentgeltlich zur Verfügung gestellt.
        
        „Textform“ meint im Folgenden ausschließlich die Übermittlung per Brief oder die elektronische Übermittlung per E-Mail.
    
	\section{Name}
	\begin{enumerate}
		\item
			Der Verein führt den Namen HedgeDoc. Er soll in das Vereinsregister eingetragen werden und führt danach den Zusatz „e.V.“.
		\item
			Der Verein hat seinen Sitz in Dortmund.
		\item
			Geschäftsjahr ist das Kalenderjahr.
	\end{enumerate}
	
	\section{Zweck}
	\begin{enumerate}
		\item
			Zweck des Vereins ist die Förderung und Verbreitung freier Software zur internetbasierten Zusammenarbeit, insbesondere der Software „HedgeDoc“. Dies dient der Förderung von freiem Wissensaustausch, Volksbildung sowie Wissenschaft und Forschung.
		\item
			Der Satzungszweck wird insbesondere verwirklicht durch:
		    \begin{enumerate}
    			\item
    			    die Förderung der Bildung, des Meinungsaustauschs und der Zusammenarbeit von Anwender:innen, Entwickler:innen und Forscher:innen,
			    \item
			        die Weiterentwicklung und Forschung an freier Software zur internetbasierten Zusammenarbeit,
		        \item
		            die Bereitstellung freier Software zur internetbasierten Zusammenarbeit, unterstützende Bilder, Töne, Daten und Dokumentation sowie Förderung deren Verfügbarkeit und die Erstellung und Verbreitung von Informationsmaterial darüber,
	            \item
	                Beiträge zur sachkundigen Information der Öffentlichkeit im Tätigkeitsbereich des Vereins und Teilnahme an Messen und Kongressen um die Informationen einem breiteren Spektrum von Anwender:innen zugänglich zu machen,
	            \item
	                der Organisation von Kongressen und allgemein zugänglichen Vorträgen zur Weiterbildung der Projektteilnehmenden und Anwender:innen
	            \item
	                das Bewahren der freien Rechte der Projektteilnehmenden zum Schutz vor kommerziellen Interessen Dritter. 
		    \end{enumerate}
	\end{enumerate}
	
	\section{Mitgliedschaft}
	\begin{enumerate}
		\item
			Es wird unterschieden zwischen:
			\begin{enumerate}
			    \item ordentlichen Mitgliedern und
			    \item Fördermitgliedern.
			\end{enumerate}
		\item
			Ordentliches Mitglied des Vereins kann jede natürliche oder juristische Person werden, welche die Ziele des Vereins unterstützt. Juristische Personen benennen dem Vorstand eine Person, die dieses Mitglied in der Mitgliederversammlung vertritt und stellvertretend Ämter in den Vereinsorganen übernimmt.
		\item
		    Fördermitglied kann jede natürliche oder juristische Person oder Personenvereinigung sein, die bereit ist, die Zwecke des Vereins ideell oder materiell zu unterstützen. 
	    \item
            Die Mitgliedschaft wird erworben aufgrund eines Beitrittsantrags in Textform, der mittels einstimmigem Beschluss des Vorstandes angenommen wird. Die Entscheidung über die Aufnahme ist den Aufnahmekandidat:innen in Textform mitzuteilen. Hauptentscheidungskriterium für die Aufnahme von ordentlichen Mitgliedern soll das von den Aufnahmekandidat:innen über einen längeren Zeitraum gezeigte Engagement und der dabei geleistete Beitrag im Sinne der Vereinsziele sein.
	    \item
            Die Mitgliedschaft endet
            \begin{enumerate}
                \item
                    durch Austritt zum Ende des Geschäftsjahres, der dem Vorstand des Vereins in Textform mindestens einen Monat vor Ende des Geschäftsjahres mitzuteilen ist.
                \item
	                \label{sec:mitgliedschaft:ende:unehrenhaft}
                    durch Ausschluss bei unehrenhaften Handlungen oder vereinsschädigendem Verhalten. Hierüber entscheidet die Mitgliederversammlung.
                \item
                    durch Tod des Mitglieds.
                \item
                    wenn das Mitglied mehr als drei Monate mit der Erfüllung der Beitragspflicht im Rückstand ist und trotz schriftlicher Mahnung durch den Vorstand unter Androhung des Ausschlusses die Rückstände nicht eingezahlt hat.
            \end{enumerate}
	\end{enumerate}
	
	\section{Rechte und Pflichten der Mitglieder}
	\begin{enumerate}
		\item
			Jedes ordentliche Mitglied hat volles Antrags- und Stimmrecht in der Mitgliederversammlung.
		\item
			Fördermitglieder haben ein Antrags- und Rederecht in der Mitgliederversammlung, jedoch kein Stimmrecht.
		\item
			Jedes stimmberechtigte Mitglied hat eine Stimme.
		\item
			Die Mitglieder haben die von der Mitgliederversammlung in der Beitragsordnung festgesetzten Beiträge zu entrichten.
		\item
			Die Mitglieder haben eine Änderung ihrer angegebenen Mitgliederdaten dem Vorstand anzuzeigen.
	\end{enumerate}

	\section{Organe des Vereins}
	\begin{enumerate}
		\item
			Der Verein hat die Organe
			\begin{enumerate}
			    \item
			        Mitgliederversammlung und
			    \item
			        Vorstand.
			\end{enumerate}
		\item
             \label{sec:organe:vorstand}
		    Der Vorstand besteht dabei aus
		    \begin{enumerate}
		        \item
		            1. Vorsitz,
	            \item
	                2. Vorsitz und
                \item
                    Kassenverwaltung.
		    \end{enumerate}
	    \item
	        Dem erweiterten Vorstand gehören zusätzlich beratend bis zu vier Beisitzende an. Diese sind zu Vorstandssitzungen einzuladen, besitzen dort jedoch kein Stimmrecht.
	\end{enumerate}
	
	\section{Vorstand}
	\begin{enumerate}
        \item
            Der Verein wird gerichtlich und außergerichtlich durch eines der in \ref{sec:organe:vorstand} genannten Mitglieder des Vorstands vertreten.
        \item
            Der Vorstand führt die Geschäfte des Vereins ehrenamtlich und beschließt über alle Vereinsangelegenheiten, soweit sie nicht eines Beschlusses der Mitgliederversammlung bedürfen. Er führt die Beschlüsse der Mitgliederversammlung aus. Der Vorstand kann sich eine Geschäftsordnung geben.
        \item 
            Die Mitgliederversammlung kann unter Berücksichtigung der Haushaltslage beschließen, dass Mitglieder des Vorstands eine pauschale Aufwandsentschädigung erhalten.
        \item 
            Der Vorstand kann Beschlüsse auch schriftlich, telefonisch oder per E-Mail, in einer Videokonferenz oder in einer gemischten Sitzung aus Anwesenden und Audio- oder Videokonferenz fassen, wenn kein Mitglied des Vorstands diesem Verfahren widerspricht.
        \item 
            Über die Sitzungen des Vorstandes ist ein Protokoll durch ein Vorstandsmitglied anzufertigen, dessen Richtigkeit durch die Unterschrift eines weiteren Vorstandsmitglieds, das in der betreffenden Sitzung anwesend war, zu bestätigen ist.
        \item 
            Beschlüsse des Vorstandes bedürfen zu ihrer Rechtswirksamkeit der einfachen Mehrheit der stimmberechtigten Mitglieder des Vorstandes.
        \item 
            Der Vorstand wird von der Mitgliederversammlung für die Dauer eines Jahres gewählt. Wiederwahl ist zulässig. Der Vorstand bleibt bis zur Wahl eines neuen Vorstandes im Amt. Wählbar sind nur Vereinsmitglieder. Scheidet ein Mitglied des Vorstands während der Amtsperiode aus, werden die Aufgaben des ausgeschiedenen Mitglieds bis zur nächsten Mitgliederversammlung von den verbleibenden Vorstandsmitgliedern übernommen.
	\end{enumerate}
	
	\section{Mitgliederversammlung}
	\begin{enumerate}
	    \item
	        \label{sec:mv:einberufung}
	        Die Mitgliederversammlung findet wenigstens einmal im Jahr statt. Sie wird in der Regel von einer Person aus dem Vorstand geleitet. Die Einladungen zur Mitgliederversammlung werden den Mitgliedern mindestens zwei Wochen vor der Mitgliederversammlung unter Angabe der Tagesordnung zugesandt. Einladungen werden üblicherweise elektronisch übermittelt.
        \item
            Die Mitgliederversammlung kann auch im Wege der elektronischen Kommunikation (z.B. per Telefon- oder Videokonferenz) oder in einer gemischten Versammlung aus Anwesenden und Telefon- oder Videokonferenz durchgeführt werden. Ob und wie die Mitgliederversammlung in einer physischen Sitzung oder im Wege der elektronischen Kommunikation oder in einer gemischten Versammlung durchgeführt wird, entscheidet der Vorstand. Die Entscheidung muss in der Einladung mitgeteilt werden.
        \item
            Die Mitgliederversammlung ist beschlussfähig, wenn sie gemäß \ref{sec:mv:einberufung} ordentlich einberufen wurde und mindestens 1/4 der stimmberechtigten Mitglieder anwesend sind.
        \item
            Im Falle der Beschlussunfähigkeit hat der Vorstand binnen einer Woche zu einer neuen Mitgliederversammlung mit gleicher Tagesordnung und Ladungsfrist einzuladen, die ohne Rücksicht auf die Präsenz der Mitglieder beschlussfähig ist.
        \item
            Die Mitgliederversammlung ist zuständig für
            \begin{enumerate}
                \item
                    Wahl und Abwahl des Vorstandes, einschließlich des erweiterten Vorstandes,
                \item
                    Entgegennahme des Rechenschaftsberichts des Vorstandes,
                \item
                    Entlastung des Vorstandes,
                \item
                    Entscheidung über die Erhebung von Beiträgen und Verabschiedung einer Beitragsordnung,
                \item
                    \label{sec:mv:satzungsänderung}
                    Satzungsänderungen (einschließlich der Änderung des Vereinszwecks),
                \item
                    \label{sec:mv:ausschluss}
                    Ausschluss von Mitgliedern nach \ref{sec:mitgliedschaft:ende:unehrenhaft} und
                \item
                    \label{sec:mv:auflösung}
                    Auflösung des Vereins.
            \end{enumerate}
        \item
            Beschlüsse nach Ziffer \ref{sec:mv:satzungsänderung}, \ref{sec:mv:ausschluss} und \ref{sec:mv:auflösung} benötigen eine Mehrheit von 3/4 der anwesenden stimmberechtigten Mitglieder und müssen in der Einladung zur Mitgliederversammlung angekündigt werden.
        \item
            Bei Wahlen ist gewählt, wer mehr als die Hälfte der abgegebenen gültigen Stimmen erhalten hat. Hat niemand mehr als die Hälfte der abgegebenen gültigen Stimmen erhalten, so findet zwischen den beiden Kandidierenden, die die meisten Stimmen erhalten haben, eine Stichwahl statt. Gewählt ist dann das Mitglied, welches die meisten Stimmen erhalten hat. Bei gleicher Stimmenzahl ist einmal ein neuer Wahlgang erforderlich; besteht die Stimmgleichheit fort, entscheidet das Los.
        \item
            Außerordentliche Mitgliederversammlungen können bei Bedarf stattfinden. Der Vorstand beruft eine außerordentliche Mitgliederversammlung von sich aus bei Vorliegen eines wichtigen Grundes ein oder wenn mindestens 1/10 aller Mitglieder dies in Textform unter Angabe eines Grundes beantragen.
        \item
            Über die Beschlüsse und den wesentlichen Verlauf der Mitgliederversammlung ist ein Protokoll anzufertigen. Dieses wird von Versammlungsleiter:in und Protokollführer:in unterschrieben. Ein:e Protokollführer:in wird von der Mitgliederversammlung auf Vorschlag der Versammlungsleitung bestimmt.
        \item
            Jedes Mitglied kann bis zum Beginn der Mitgliederversammlung Anträge auf Ergänzung der Tagesordnung bei der Versammlungsleitung stellen.
        \item
            Jedes Mitglied kann sich bei der Beschlussfassung durch ein anderes Mitglied vertreten lassen. Die Vollmacht ist vor Beginn der Mitgliederversammlung in Textform der Versammlungsleitung zu übermitteln.
	\end{enumerate}
	
	\section {Beitragsordnung}
	\begin{enumerate}
	    \item
	        Die Beitragsordnung definiert Höhe und Regeln zur Entrichtung der Mitgliedsbeiträge. Sie ist für alle Mitglieder verbindlich und nicht Teil der Satzung.
	    \item
	        Die Beitragsordnung wird den Mitgliedern zusammen mit der Satzung zur Verfügung gestellt.
		\item
		    Neue Versionen der Beitragsordnung gelten ab Veröffentlichung.
	    \item
	        Die Veröffentlichung einer neuen Version der Beitragsordnung durch den Vorstand muss maximal zwei Wochen nach Beschluss der Mitgliederversammlung erfolgen.
	\end{enumerate}
	
	\section{Auflösung des Vereins}
    \begin{enumerate}
        \item 
            Der Verein wird durch Beschluss der Mitgliederversammlung gemäß \ref{sec:mv:auflösung} oder aus gesetzlichen Gründen aufgelöst.
        \item
            Bei Auflösung des Vereins erfolgt keine Rückgewähr des Vereinsvermögens an die Mitglieder des Vereins.
        \item
            Bei Auflösung oder Aufhebung des Vereins fällt das Vermögen des Vereins an eine juristische Person oder eine andere Körperschaft zwecks Verwendung zur Förderung von freier Software, welche von der Mitgliederversammlung mit 3/4-Mehrheit bestimmt wird.
    \end{enumerate}
	
	\section{Salvatorische Klausel}
    \begin{enumerate}
        \item
            Sollte eine der Bestimmungen dieser Satzung ganz oder teilweise rechtswidrig oder unwirksam sein oder werden, so wird die Gültigkeit der übrigen Bestimmungen dadurch nicht berührt. In einem solchen Fall ist die Satzung vielmehr ihrem Sinne gemäß zur Durchführung zu bringen. Beruht die Ungültigkeit auf einer Leistungs- oder Zeitbestimmung, so tritt an ihre Stelle das gesetzlich zulässige Maß.
        \item
            Die rechtswidrige oder unwirksame Bestimmung ist unverzüglich durch Beschluss der nächsten Mitgliederversammlung zu ersetzen.
    \end{enumerate}
\end{document}
